
\chapter{Podsumowanie} 

Rezultatem pracy inżynierskiej są dwa moduły wspomagające analizę i przetwarzanie danych w gabinetach stomatologicznych. Pierwszym modułem jest system oznaczania zębów. Zgodnie z poczynionymi założeniami, system pozwala na:
\begin{itemize}
    \item Utworzenie nowej kartoteki pacjenta lub edycję już istniejącej
    \item Wyświetlanie i edycje danych uzębienia pacjenta za pomocą diagramu zębowego
    \item Wprowadzanie danych do diagramu za pomocą interfejsu głosowego
    \item Eksport diagramu na potrzeby pacjenta
\end{itemize}
Aplikacja pierwszego modułu składa się z klienta i serwera z bazą danych. Klienta można uruchomić za pomocą  najnowszych przeglądarek internetowych. Serwer uruchamiany jest na lokalnej maszynie za pomocą konsoli systemowej.
\newline Drugi moduł dotyczący analizy odchylenia toru odwodzenia żuchwy pozwala na:
\begin{itemize}
    \item Wybranie pliku z lokalnego systemu plików.
    \item Wyświetlenie filmu podczas analizy z zaznaczonymi obliczanymi kątami.
    \item Wyświetlenie maksymalnego odchylenia żuchwy podczas jej odwodzenia w wartościach względnych i bezwzględnych.
    \item Wyświetlenie pełnej analizy w postaci wykresu.
    \item Wybranie fragmentu filmu do analizy.
    \item Zapis wyników, wykresów i zrzutów ekranu w formacie pdf.
\end{itemize}
Tu o exceku ale nie wiem jak to opisać.


Zakończenie pracy zwane również Uwagami końcowymi lub Podsumowaniem powinno zawierać ustosunkowanie
się autora do zadań wskazanych we wstępie do pracy, a w szczególności do celu i zakresu pracy oraz
porównanie ich z faktycznymi wynikami pracy. Podejście takie umożliwia jasne określenie stopnia
realizacji założonych celów oraz zwrócenie uwagi na wyniki osiągnięte przez autora w ramach jego
samodzielnej pracy.

Integralną częścią pracy są również dodatki, aneksy i załączniki zawierające stworzone w ramach pracy programy, aplikacje i projekty.

\section{Rezultaty}
\section{Rozwój i przyszłość}